% Blueprint content for Balanced Vector Optimization
% This file contains the formal statement and can be included in a larger paper

\documentclass[a4paper,11pt]{article}

\usepackage{amsmath,amssymb,amsthm}
\usepackage{hyperref}
\usepackage{xcolor}

% Blueprint-specific packages
\usepackage{blueprint}

% Theorem environments
\newtheorem{theorem}{Theorem}[section]
\newtheorem{lemma}[theorem]{Lemma}
\newtheorem{definition}[theorem]{Definition}
\newtheorem{corollary}[theorem]{Corollary}

% Custom commands
\newcommand{\ZZ}{\mathbb{Z}}
\newcommand{\QQ}{\mathbb{Q}}
\newcommand{\NN}{\mathbb{N}}
\newcommand{\ee}{\mathbf{e}}

\title{Balanced Vector Optimization\\[0.5em]
\large Lean 4 Formalization Blueprint}
\author{}
\date{}

\begin{document}
\maketitle

\section{Introduction}

This document provides a formal blueprint for the Lean 4 formalization of the following result:
symmetric log-concave functions on weak compositions achieve their maximum on balanced vectors
and their minimum on concentrated vectors.

This abstract theorem captures the key structure used in the proof that descendant integrals
$\langle \tau_{e_1} \cdots \tau_{e_n} \rangle_g$ on moduli spaces of curves $\overline{\mathcal{M}}_{g,n}$
are maximized for balanced exponent vectors and minimized for concentrated ones.

\section{Definitions}

\begin{definition}[Weak Composition]\label{def:weak_composition}
\lean{WeakComposition}\leanok
A \emph{weak composition} of $d$ into $n$ parts is a vector $\ee = (e_1, \ldots, e_n) \in \ZZ^n$
such that $\sum_{i=1}^n e_i = d$ and $e_i \geq 0$ for all $i$.

We denote the set of such compositions by $E(n, d)$.
\end{definition}

\begin{definition}[Balanced Vector]\label{def:balanced}
\lean{IsBalanced}\leanok
A vector $\ee \in \ZZ^n$ is \emph{balanced} if $|e_i - e_j| \leq 1$ for all $1 \leq i, j \leq n$.

Equivalently: $e_i \leq e_j + 1$ and $e_j \leq e_i + 1$ for all pairs $i, j$.
\end{definition}

\begin{definition}[Concentrated Vector]\label{def:concentrated}
\lean{IsConcentrated}\leanok
A vector $\ee \in \ZZ^n$ is \emph{concentrated} at index $k$ if $e_k = d$ and $e_i = 0$ for all $i \neq k$.

In other words, $\ee = d \cdot \delta_k$ where $\delta_k$ is the $k$-th standard basis vector.
\end{definition}

\begin{definition}[Symmetric Function]\label{def:symmetric}
\lean{IsSymmetric}\leanok
A function $D : E(n,d) \to \QQ$ is \emph{symmetric} (or $S_n$-invariant) if
\[
D(\ee \circ \sigma^{-1}) = D(\ee)
\]
for all permutations $\sigma \in S_n$ and all $\ee \in E(n,d)$.
\end{definition}

\begin{definition}[Log-Concavity Condition]\label{def:log_concave}
\lean{SatisfiesLogConcavity}\leanok
A function $D : E(n,d) \to \QQ$ satisfies the \emph{log-concavity condition} if for all
$\ee \in E(n,d)$ and distinct indices $i \neq j$ with $e_i \geq 1$ and $e_j \geq 1$:
\[
D(\ee)^2 \geq D(\ee - \delta_i + \delta_j) \cdot D(\ee + \delta_i - \delta_j)
\]
where $\delta_i$ denotes the $i$-th standard basis vector.
\end{definition}

\begin{definition}[Strict Positivity]\label{def:positive}
\lean{IsStrictlyPositive}\leanok
A function $D : E(n,d) \to \QQ$ is \emph{strictly positive} if $D(\ee) > 0$ for all $\ee \in E(n,d)$.
\end{definition}

\begin{definition}[Symmetric Log-Concave Function]\label{def:slc_function}
\lean{SymmetricLogConcaveFunction}\leanok
\uses{def:symmetric, def:log_concave, def:positive}
A function $D : E(n,d) \to \QQ$ is a \emph{symmetric log-concave function} if it satisfies:
\begin{enumerate}
    \item $S_n$-symmetry (Definition~\ref{def:symmetric})
    \item The log-concavity condition (Definition~\ref{def:log_concave})
    \item Strict positivity (Definition~\ref{def:positive})
\end{enumerate}
\end{definition}

\section{Main Results}

\begin{theorem}[Unimodality of Log-Concave Palindromic Sequences]\label{thm:unimodal}
\lean{unimodal_of_logconcave_palindromic}\leanok
Let $s : \ZZ \to \QQ$ be a sequence that is:
\begin{itemize}
    \item Positive on $[0, q]$
    \item Log-concave on $[0, q]$: $s(t)^2 \geq s(t-1) \cdot s(t+1)$ for $1 \leq t \leq q-1$
    \item Palindromic on $[0, q]$: $s(t) = s(q - t)$ for $0 \leq t \leq q$
\end{itemize}
Then $s$ is unimodal on $[0, q]$:
\begin{itemize}
    \item $s(t) \leq s(t+1)$ for $t < q/2$ (increasing in first half)
    \item $s(t) \leq s(t-1)$ for $t > q/2$ (decreasing in second half)
\end{itemize}
\end{theorem}

\begin{theorem}[Maximum on Balanced Vectors]\label{thm:max}
\lean{SymmetricLogConcaveFunction.maximized_on_balanced}\leanok
\uses{def:slc_function, def:balanced, thm:unimodal}
Let $D : E(n,d) \to \QQ$ be a symmetric log-concave function. Then for any $\ee \in E(n,d)$,
there exists a balanced vector $\ee' \in E(n,d)$ such that
\[
D(\ee) \leq D(\ee').
\]
In particular, $D$ achieves its maximum on a balanced vector.
\end{theorem}

\begin{theorem}[Minimum on Concentrated Vectors]\label{thm:min}
\lean{SymmetricLogConcaveFunction.minimized_on_concentrated}\leanok
\uses{def:slc_function, def:concentrated, thm:unimodal}
Let $n > 0$, $d \geq 0$, and $D : E(n,d) \to \QQ$ be a symmetric log-concave function.
Then for any $\ee \in E(n,d)$, there exists a concentrated vector $\ee' \in E(n,d)$ such that
\[
D(\ee') \leq D(\ee).
\]
In particular, $D$ achieves its minimum on a concentrated vector.
\end{theorem}

\begin{theorem}[Main Theorem]\label{thm:main}
\lean{main_theorem}\leanok
\uses{thm:max, thm:min}
\proves{thm:main}
Let $n > 0$, $d \geq 0$, and let $D : E(n,d) \to \QQ$ be a symmetric log-concave function. Then:
\begin{enumerate}
    \item[\textup{(a)}] \textbf{Maximum:} $D$ achieves its maximum on balanced vectors.
    \item[\textup{(b)}] \textbf{Minimum:} $D$ achieves its minimum on concentrated vectors.
\end{enumerate}
\end{theorem}

\section{Application to Descendant Integrals}

The main application is to the function $D : E(n, 3g-3+n) \to \QQ$ defined by
\[
D(\ee) = \langle \tau_{e_1} \cdots \tau_{e_n} \rangle_g = \int_{\overline{\mathcal{M}}_{g,n}} \psi_1^{e_1} \cdots \psi_n^{e_n}
\]
where $\psi_i$ are the cotangent line classes on the moduli space of stable curves.

This function satisfies:
\begin{itemize}
    \item \textbf{Symmetry:} The symmetric group $S_n$ acts on $\overline{\mathcal{M}}_{g,n}$ by permuting
    the marked points, and $\psi_i \mapsto \psi_{\sigma(i)}$ under this action.
    \item \textbf{Log-concavity:} Follows from the Khovanskii--Teissier inequalities, since the
    $\psi$-classes are nef.
    \item \textbf{Positivity:} The $\psi$-classes are effective, and top intersections of nef
    classes on a projective variety are positive.
\end{itemize}

Therefore, Theorem~\ref{thm:main} applies, yielding:

\begin{corollary}[Extremal Descendants]
The descendant integral $\langle \tau_{e_1} \cdots \tau_{e_n} \rangle_g$ is:
\begin{itemize}
    \item Maximized when $(e_1, \ldots, e_n)$ is balanced
    \item Minimized when $(e_1, \ldots, e_n)$ is concentrated
\end{itemize}
\end{corollary}

\bibliographystyle{alpha}
\bibliography{references}

\end{document}
