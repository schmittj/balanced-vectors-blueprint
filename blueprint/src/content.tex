\chapter{Balanced Vector Optimization}

\section{Introduction}

This blueprint documents the Lean 4 formalization of a theorem about symmetric log-concave functions on weak compositions. The main result states that such functions are maximized on balanced vectors and minimized on concentrated vectors.

The primary application is to descendant integrals on moduli spaces of curves:
\[
\langle \tau_{e_1} \cdots \tau_{e_n} \rangle_g = \int_{\overline{\mathcal{M}}_{g,n}} \psi_1^{e_1} \cdots \psi_n^{e_n}
\]

\section{Definitions}

\begin{definition}[Weak Composition]\label{def:weak_composition}
\lean{WeakComposition}
\leanok
A \emph{weak composition} of $d$ into $n$ parts is an $n$-tuple $e = (e_1, \ldots, e_n) \in \mathbb{Z}^n$ such that:
\begin{itemize}
\item $\sum_{i=1}^n e_i = d$
\item $e_i \geq 0$ for all $i$
\end{itemize}
We denote the set of such tuples by $E(n,d)$.
\end{definition}

\begin{definition}[Balanced Vector]\label{def:balanced}
\lean{IsBalanced}
\leanok
\uses{def:weak_composition}
A weak composition $e \in E(n,d)$ is \emph{balanced} if $|e_i - e_j| \leq 1$ for all $i, j \in \{1, \ldots, n\}$.
\end{definition}

\begin{definition}[Concentrated Vector]\label{def:concentrated}
\lean{IsConcentrated}
\leanok
\uses{def:weak_composition}
A weak composition $e \in E(n,d)$ is \emph{concentrated} if there exists $k$ such that $e = d \cdot \delta_k$, i.e., all mass is on a single index.
\end{definition}

\begin{definition}[Symmetric Function]\label{def:symmetric}
\lean{IsSymmetric}
\leanok
\uses{def:weak_composition}
A function $D : E(n,d) \to \mathbb{Q}$ is \emph{$S_n$-symmetric} if $D(e \circ \sigma^{-1}) = D(e)$ for all permutations $\sigma \in S_n$.
\end{definition}

\begin{definition}[Log-Concave Function]\label{def:logconcave}
\lean{SatisfiesLogConcavity}
\leanok
\uses{def:weak_composition}
A function $D : E(n,d) \to \mathbb{Q}$ is \emph{log-concave} if for all $e \in E(n,d)$ and indices $i \neq j$ with $e_i \geq 1$ and $e_j \geq 1$:
\[
D(e)^2 \geq D(e - \delta_i + \delta_j) \cdot D(e + \delta_i - \delta_j)
\]
(The conditions $e_i \geq 1$ and $e_j \geq 1$ ensure both modified vectors remain in $E(n,d)$.)
\end{definition}

\begin{definition}[Symmetric Log-Concave Function]\label{def:symmetric_logconcave_function}
\lean{SymmetricLogConcaveFunction}
\leanok
\uses{def:weak_composition, def:symmetric, def:logconcave}
A \emph{symmetric log-concave function} on $E(n,d)$ is a function $D : E(n,d) \to \mathbb{Q}$ satisfying:
\begin{enumerate}
\item $S_n$-symmetry
\item Log-concavity
\item Strict positivity: $D(e) > 0$ for all $e$
\end{enumerate}
\end{definition}

\section{Main Result}

\begin{theorem}[Main Theorem]\label{thm:main}
\lean{main_theorem_paper}
\leanok
\uses{def:symmetric_logconcave_function, def:balanced, def:concentrated, thm:maximized_on_balanced, thm:minimized_on_concentrated}
Let $D : E(n,d) \to \mathbb{Q}$ be a function satisfying:
\begin{enumerate}
\item \textbf{$S_n$-symmetry:} $D(e \circ \sigma^{-1}) = D(e)$ for all permutations $\sigma$
\item \textbf{Log-concavity:} $D(e)^2 \geq D(e - \delta_i + \delta_j) \cdot D(e + \delta_i - \delta_j)$ for all $i \neq j$ with $e_i, e_j \geq 1$
\item \textbf{Strict positivity:} $D(e) > 0$ for all $e$
\end{enumerate}
Then:
\begin{itemize}
\item \textbf{Maximum:} There exists a balanced vector $b \in E(n,d)$ such that $D(e) \leq D(b)$ for all $e \in E(n,d)$.
\item \textbf{Minimum:} There exists a concentrated vector $c \in E(n,d)$ such that $D(c) \leq D(e)$ for all $e \in E(n,d)$.
\end{itemize}
\end{theorem}

\section{Key Lemmas and Theorems}

\begin{theorem}[Maximized on Balanced]\label{thm:maximized_on_balanced}
\lean{exists_balanced_maximizer}
\leanok
\uses{def:symmetric_logconcave_function, def:balanced, lem:balanced_maximizes}
Let $D$ be a symmetric log-concave function on $E(n,d)$. Then there exists a balanced vector $b \in E(n,d)$ such that $D(e) \leq D(b)$ for all $e \in E(n,d)$.
\end{theorem}

\begin{theorem}[Minimized on Concentrated]\label{thm:minimized_on_concentrated}
\lean{exists_concentrated_minimizer}
\leanok
\uses{def:symmetric_logconcave_function, def:concentrated, lem:concentrated_minimizes}
Let $D$ be a symmetric log-concave function on $E(n,d)$. Then there exists a concentrated vector $c \in E(n,d)$ such that $D(c) \leq D(e)$ for all $e \in E(n,d)$.
\end{theorem}

\begin{lemma}[Unimodal of Log-Concave Palindromic]\label{lem:unimodal}
\lean{unimodal_of_logconcave_palindromic}
\leanok
If a sequence is log-concave and palindromic, then it is unimodal (non-decreasing up to the middle, then non-increasing).
\end{lemma}

\begin{lemma}[Balanced Maximizes (Pointwise)]\label{lem:balanced_maximizes}
\lean{balanced_maximizes}
\leanok
\uses{def:balanced, lem:unimodal}
For a symmetric log-concave function $D$ and any $e \in E(n,d)$, there exists a balanced $b \in E(n,d)$ with $D(e) \leq D(b)$.
\end{lemma}

\begin{lemma}[Concentrated Minimizes (Pointwise)]\label{lem:concentrated_minimizes}
\lean{concentrated_minimizes}
\leanok
\uses{def:concentrated, lem:unimodal}
For a symmetric log-concave function $D$ and any $e \in E(n,d)$, there exists a concentrated $c \in E(n,d)$ with $D(c) \leq D(e)$.
\end{lemma}
